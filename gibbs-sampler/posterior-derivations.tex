\documentclass[a4paper,12pt]{article} % Document class

% Packages for additional functionality
\usepackage[utf8]{inputenc}  % Encoding
\usepackage{amsmath, amssymb} % Math symbols
\usepackage{graphicx}         % Images
\usepackage{hyperref}         % Clickable links
\usepackage{xcolor}           % Color text
\usepackage{geometry}         % Page layout
\usepackage{bm}
\geometry{margin=1in}        % Set 1-inch margins
\newcommand{\bx}{\bm{x}}
\newcommand{\by}{\bm{y}}
\newcommand{\bM}{\bm{M}}
\newcommand{\bQ}{\bm{Q}}
\newcommand{\bI}{\bm{I}}
\newcommand{\bw}{\bm{w}}
\newcommand{\bv}{\bm{v}}

% Document title, author, and date
\title{Deriving the posterior}
\date{\today} % Automatically inserts today's date
\maketitle

\begin{document}


Start by considering a linear Gaussian state-space model:
\begin{align*}
\bx_t &= (\bM_1 + a\bM_2)\bx_{t - 1} + \bw_t\\
\by_t &= \bx_t + \bv_t,
\end{align*}
where $\bw_t \sim N(0, \bQ)$ and $\bv_t \sim N(0, \tau^2\bI)$ and $t = 1, \dots, T$. We observe $\by_t$ at each $t$ while $\bx_t, \bw_t, \bv_t$ are latent. We want to calculate the posterior of $a$ assuming, that $p(a)$ is the prior for $a$ We assume, for simplicity, that $\tau$ and $\bQ$ are known. Formally, we need to calculate
\begin{equation}
  p(a|\bx_{1:T}, \by_{1:T}) = \frac{p(a, \bx_{1:T}, \by_{1:T})}{p(\bx_{1:T}, \by_{1:T})}. 
\end{equation}
Finding the denominator is difficulut be we might not really need it. Let's first start with the numerator.
\begin{multline}
  p(a, \bx_{1:T}, \by_{1:T}) = p(\by_{1:T} | \bx_{1:T}, a) p(\bx_{1:T}|a) p(a)
  \end{multline}


\end{document}
