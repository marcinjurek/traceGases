\documentclass[a4paper,12pt]{article} % Document class

% Packages for additional functionality
\usepackage[utf8]{inputenc}  % Encoding
\usepackage{amsmath, amssymb} % Math symbols
\usepackage{graphicx}         % Images
\usepackage{hyperref}         % Clickable links
\usepackage{xcolor}           % Color text
\usepackage{geometry}         % Page layout
\usepackage{bm}
\geometry{margin=1in}        % Set 1-inch margins
\newcommand{\bx}{\bm{x}}
\newcommand{\by}{\bm{y}}
\newcommand{\bM}{\bm{M}}
\newcommand{\bQ}{\bm{Q}}
\newcommand{\bI}{\bm{I}}
\newcommand{\bw}{\bm{w}}
\newcommand{\bv}{\bm{v}}
\newcommand{\bd}{\bm{d}}

% Document title, author, and date
\title{Deriving the posterior}
\date{\today} % Automatically inserts today's date

\begin{document}
\maketitle

Start by considering a linear Gaussian state-space model:
\begin{align*}
\bx_t &= (\bM_1 + a\bM_2)\bx_{t - 1} + \bw_t\\
\by_t &= \bx_t + \bv_t,
\end{align*}
where $\bw_t \sim N(0, \bQ)$ and $\bv_t \sim N(0, \tau^2\bI)$ and $t = 1, \dots, T$. We observe $\by_t$ at each $t$ while $\bx_t, \bw_t, \bv_t$ are latent. We want to calculate the posterior of $a$ assuming, that $\pi(a)$ is the prior for $a$ We assume, for simplicity, that $\tau$ and $\bQ$ are known. Formally, we need to calculate
\begin{equation}
  p(a|\bx_{1:T}, \by_{1:T}) = \frac{p(a, \bx_{1:T}, \by_{1:T})}{p(\bx_{1:T}, \by_{1:T})}. 
\end{equation}
Finding the denominator is difficulut be we might not really need it. Let's first start with the numerator.
$$
p(a, \bx_{1:T}, \by_{1:T}) = p(\by_{1:T} | \bx_{1:T}, a) p(\bx_{1:T}|a) \pi(a).
$$
For the sake of this derivation the first term can be considered constant (we'll call it $C$) while we can expand the second:
\begin{multline*}
  p(a, \bx_{1:T}, \by_{1:T}) = Cp(\bx_1)\prod_{t=2}^Tp(\bx_t|\bx_{t-1}, a) \pi(a) = \\
  Cp(\bx_1)\pi(a) \prod_{t=2}^T \frac{1}{\sqrt{2\pi|\bQ|}} \exp\left(-\frac{1}{2}\bd(a)'\bQ^{-1}\bd(a)\right),
\end{multline*}
where $\bd(a) = \bx_t - (\bM_1 + a\bM_2)\bx_{t-1}$.
Taking logs, and suppressing explicit dependence on $\bx_{1:T}, \by_{1:T}$ for brevity, we get
$$
\log p(a) = \log C + \log(p(\bx_1)) + \log \pi(a) - \frac{T}{2}\log (2\pi |\bQ|) - \frac{1}{2}\sum_{t=2}^T \bd(a)'\bQ^{-1}\bd(a).
$$
Since we are only interested in the terms that include $a$ we can collect the other terms and call them $B$. This gives us
$$
\log p(a) = B + \log \pi(a) -\frac{1}{2} \sum_{t=2}^T \left(\bx_t - (\bM_1 + a\bM_2)\bx_{t-1}\right)'\bQ^{-1} \left(\bx_t - (\bM_1 + a\bM_2)\bx_{t-1}\right).
$$
We can now break up the quadratic form under the summation so that we are left with only the terms that include $a$. We thus call all the terms that don't involve $a$ $A$
\begin{multline}
  \log p(a) = A + \log \pi(a) +\frac{a}{2} \sum_{t=2}^T 2\left(\bM_2\bx_{t-1}\right)'\bQ^{-1}\left(\bx_t - \bM_1\bx_{t-1}\right) \\
  -\frac{a^2}{2}\sum_{t=2}^T \bx_{t-1}'\bM_2'\bQ^{-1}\bM_2\bx_{t-1}
\end{multline}
If we now call the sum, including the $0.5$ factors, $S_1$ and $S_2$, respectively, we can rewrite this as
$$
\log p(a) = A + \log \pi(a) + a S_1 - a^2 S_2.  
$$
Then we can complete the square to get
$$
\log p(a) = A + \log \pi(a) - S_2\left(a - \frac{S_1}{2S_2}\right)^2 + c = A+c + \log\pi(a) - \frac{1}{2\frac{1}{2S_2}}\left(a - \frac{S_1}{2S_2}\right)^2,
$$
where $c$ is some expression which does not depend on $a$ and is the result of completing the square.
If we now assume that the prior on $a$ is Gaussian, i.e. that $\log \pi(a) = -\frac{1}{2\nu^2}(a - \mu_a)^2 + \log \frac{1}{\sqrt{2\pi\nu^2}}$, we can write
$$
\log p(a) = \tilde{A} -\frac{1}{2\nu^2}(a - \mu_a)^2 - \frac{1}{2\frac{1}{2S_2}}\left(a - \frac{S_1}{2S_2}\right)^2,
$$
where $\tilde{A}$ is yet another constant which does not depend on $a$.

Combining the two sqaures and using yet another constant, $\hat{A}$ we can then write the posterior distribution of $a$ as
$$
\log p(a) = \hat{A} - \frac{1}{2\hat{\sigma}^2_a} \left(a - \hat{\mu}_a\right)^2,
$$
where
\begin{align*}
  \hat{\sigma}^2 &= \frac{1}{\frac{1}{\nu^2} + 2S_2} \\
  \hat{\mu}_a &= \frac{\frac{\mu_a}{\nu^2} +S_1}{\frac{1}{\nu^2} + 2S_2},
\end{align*}
with
\begin{align}
  S_1 &= \sum_{t=2}^T \left(\bM_2\bx_{t-1}\right)'\bQ^{-1}\left(\bx_t - \bM_1\bx_{t-1}\right)\\
  S_2 &= \frac{1}{2}\sum_{t=2}^T \bx_{t-1}'\bM_2'\bQ^{-1}\bM_2\bx_{t-1}
  \end{align}

\end{document}
